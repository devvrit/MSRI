\section{Correlation Clustering}
%Consider $G = ([n],\binom{[n]}{2})$, the complete graph on $n$ vertices. Each edge in $G$ is associated with a label, either $+$ or $-$. Two vertices are said to be similar if the edge between them is $+$ and dissimilar if the edge between them is $-$. The cost of a clustering $C$ over $[n]$ is defined as the sum of the number of similar edges between vertices in different clusters, and dissimilar vertices between vertices in the same cluster. The cost of a clustering $C$ over a subset of vertices $V \subseteq [n]$ is computed by only counting the cost incurred by the induced subgraph of $V$. For a clustering $C$, this is denoted $\cost (C, V)$.
\rnote{Use ``\Cref'' instead of ``\ref'' as it will smartly append a prefix of Theorem, or Lemma, or whatever. We don't need to do it.}
\rnote{Use ``{\cal I}'' for instances as it is somewhat standard.}

The classical problem is correlation clustering can be stated as follows.
\begin{problem}[Correlation Clustering] \label{prob:1}
We are given a complete graph $G = (V,E)$, and a labelling of each edge as either positive or negative, denoting whether the two points are similar or dissimilar. The goal is to compute a partition ${\cal C} = \{C_1, C_2, \ldots, C_r\}$ of $V$ (so $V = \dot\cup_{1 \leq i \leq r} C_i$ is a disjoint union of the $C_i$'s) to minimize the \emph{cost} of the clustering, which is the sum of positive edges whose end-points are in different clusters and the negative edges whose end-points are in the same cluster. 
%	The correlation clustering problem is to cluster the vertices in $[n]$ into as many or as few clusters as possible such that the total cost is minimized.
%	\begin{equation*}
%		\text{Find } C^* = \argmin_C \cost (C, [n]).
%	\end{equation*}
\end{problem}

In this paper, we consider the \emph{robust correlation clustering} problem (\robcc) which is a generalization of the correlation clustering problem with the added goal of being resilient to \emph{outliers} in the given data. 

\begin{problem}[Robust Correlation Clustering] \label{prob:2}
We are given a complete graph $G = (V,E)$, and a labelling of each edge as either positive or negative, denoting whether the two points are similar or dissimilar. We are also given a parameter $m$, which denotes the number of points we can discard while clustering. The goal is to identify a set $D \subseteq V$ of \emph{outliers} of size $m$, and cluster the remaining points $V\setminus D$ to minimize the cost of the clustering. That is, we would like to find a partition ${\cal C} = \{C_1, C_2, \ldots, C_r\}$ of $V \setminus D$ to minimize the sum of positive edges whose end-points are in different clusters in ${\cal C}$ and the negative edges whose end-points are in the same cluster in ${\cal C}$. 
\end{problem}

In addition, we also consider the \robcc problem on general graphs, where the edge-set is not the complete graph over $V$, to capture the notion of missing similarity information between certain pairs of points. This naturally extends the analogous study of correlation clustering in general graphs \cite{XXX}.

\subsection{Our Results}
\begin{theorem} \label{thm:complete-hardness}
It is NP-hard to obtain any bi-criteria $(a,b)$-approximation algorithm for \robcc on complete graphs when $b < \alpha_{\sf VertexCover}$.
\end{theorem}


\begin{theorem} \label{thm:complete1}
There is an efficient bi-criteria $(a,b)$-approximation algorithm for \robcc on complete graphs.
\end{theorem}

\begin{theorem} \label{thm:complete2}
There is an efficient and \emph{purely combinatorial} bi-criteria $(a,b)$-approximation algorithm for \robcc on complete graphs.
\end{theorem}

\begin{theorem} \label{thm:general-hardness}
It is NP-hard to obtain any bi-criteria $(a,b)$-approximation algorithm for \robcc on general graphs for $b < \alpha_{\sf MultiCut}$ or $a < \alpha_{\sf MultiCut}$.
\end{theorem}

\begin{theorem} \label{thm:general}
There is an efficient bi-criteria $(a,b)$-approximation algorithm for \robcc on general graphs.
\end{theorem}
